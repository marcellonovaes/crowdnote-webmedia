The method used for the distribution of jobs among workers was defined according to two main criteria: 1. In each task, a worker can not contribute more than once to the same item; 2. It is desirable to obtain homogeneous coverage for items with similar amounts of contributions.

To meet the first requirement, the fingerprint of each worker is captured. This fingerprint is generated from the IP address and the signature of the Web browser. In this way, the system does not send the worker a job with an item he has already contributed.

The solution found to achieve a relatively homogeneous coverage of the items to be annotated, was adopted a routing strategy based on FIFO (first in first out) and LIFO (last in first out) structures, in accordance with the following rules:


\begin{itemize}

\item For each task, initially the items to be annotated are inserted into the entry LIFO.

\item For each job request, items are removed from the entry LIFO until an item that has not yet been annotated by the worker is found.

\item Items that have already been annotated by the worker are inserted into a temporary FIFO, which are re-entered into the entry LIFO when is found an item that still was not annotated by the worker.

\item When a worker has already annotated all the items, he is directed to a thank you page and receives no more items for that task.

\item Items taken from the entry LIFO and separated to be annotated, are inserted into the exit LIFO.

\item When the entry LIFO is empty, it is replenished by the exit LIFO, preserving an original order as far as possible.

\end{itemize}

This strategy aims to avoid that an item already annotated by a worker needs to wait until the next round is marked by another contributor. Initially, the use of a circular FIFO was considered. However, in a crowdsourcing environment, it would be very expensive to create mechanisms so that the elements already annotated by one worker would not sink until the end of the FIFO, losing the chance of being annotated by another worker in that round.
