
Annotation is a term used to refer to both the annotation process and the product generated by the annotation. These annotations are metadata related to the objects you want to describe in relation to certain aspects. Siegfried Handschuh presented in 2006 some interesting definitions of semantic annotation \cite{oren2006semantic}, among them:

\begin{quote}
"An annotation A is a tuple (as, ap, ao, ac), where \textbf{as} is the subject of the annotation (the annotated data) \textbf{ao} is the object of the annotation (the annotating data) \textbf{ap} is the predicate (the annotation relation) that defines the type of relationship between as and ao, and \textbf{ac} is the context in which the annotation is made."
\end{quote}

These annotations  provide understanding about different types of artifacts \cite{Singhal:2014:GSA:2611040.2611056} such text, image, video, among others, and can be generated manually or automatically \cite{Mihalcea:2007:WLD:1321440.1321475}, so that you can use the most suitable strategy for each scenario. They are used for classification, indexing and other applications that are intended to facilitate the work of both users and systems that use annotated items. Annotations are the basis for semantics web \cite{Cimiano:2004:TSW:988672.988735}, as well as for search engines and information retrieval systems \cite{Datta:2008:IRI:1348246.1348248,Junior:2008:MMS:1809980.1810011,Dmitriev:2006:UAE:1135777.1135900}. 




