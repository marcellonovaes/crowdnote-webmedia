This paper presented CrowdNote, a crowdsourcing environment that can achieve complex video annotation from a crowd of untrained and nonspecializing contributors. CrowdNote can be used as template for different kinds of crowdsourcing applications based on video annotation.

To demonstrate how CrowdNote works, was created an instance of it that consists of a video enrichment application. This application used the crowd to annotate the points of interest present in the video, collect extra content to associate with them, select the collected content, and positionate the extra content in the video when each point of interest happens.

However, various types of applications can be generated as instances of CrowdNote. It is possible to create applications to annotate multiple aspects of scenes, generate transcriptions, human translations, among others.

The most interesting aspect of CrowdNote is that it offers a way to make complex video annotations without the work of experts, without the need to create expensive and sophisticated annotation systems, or ask employees to perform difficult and laborious tasks.

Traditional crowdsourcing approaches are struggling to achieve these complex annotations, but the strategy of dividing the problem into microtasks that collect simple annotations and cascades them to generate complex annotations proved to be functional.

Future versions of CrowdNote will incorporate features to assist the owner in generating bootstrap input, as well as selecting aggregation methods and annotation tools from a sample library.

The CrowdNote instance built for this work can be freely download, used and modified from GitHub [Removed for double-blind review] . 

